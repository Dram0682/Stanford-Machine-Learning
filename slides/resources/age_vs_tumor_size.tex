\documentclass{article}
\usepackage{graphicx}
\usepackage[utf8]{inputenc}
\usepackage{amsmath, amssymb, latexsym}
 
\usepackage{pgfplots}
\usepackage{tikz}
\usepackage{nicefrac}
\pgfplotsset{every axis legend/.append style={
at={(0,0)},
anchor=north east}}
\usetikzlibrary{shapes,positioning,intersections,quotes}

\definecolor{darkgreen}{rgb}{0.0, 0.6, 0.0}
\definecolor{darkred}{rgb}{0.7, 0.0, 0.0}
\begin{document}

\begin{tikzpicture}
  \begin{axis}[
      axis x line=middle,
      axis y line=middle,
      width=12cm, height=5cm,     % size of the image
      grid = none,
      grid style={dashed, gray!0},
      %xmode=log,log basis x=10,
      %ymode=log,log basis y=10,
      xmin=0,     % start the diagram at this x-coordinate
      xmax= 1,    % end   the diagram at this x-coordinate
      ymin=0,     % start the diagram at this y-coordinate
      ymax=1,   % end   the diagram at this y-coordinate
      %/pgfplots/xtick={0,1,...,60}, % make steps of length 5
      %extra x ticks={23},
      %extra y ticks={0, 1},
      axis background/.style={fill=white},
      ylabel=Age,
      xlabel=Tumor size,
      xticklabels={,,},
      yticklabels={,,},
      tick align=outside,
      tension=0.08]
    % plot the stirling-formulae
    \fill[blue] (5, 10) circle (5pt);
    \fill[blue] (20, 10) circle (5pt);
    \fill[blue] (30, 10) circle (5pt);
    \fill[blue] (40, 10) circle (5pt);
    \fill[blue] (45, 10) circle (5pt);
    \fill[blue] (10, 30) circle (5pt);
    \fill[blue] (25, 30) circle (5pt);
    \fill[blue] (35, 30) circle (5pt);
    \fill[blue] (15, 70) circle (5pt);
    \fill[blue] (25, 60) circle (5pt);
    \fill[blue] (15, 50) circle (5pt);
    \fill[blue] (20, 50) circle (5pt);
    \fill[blue] (30, 50) circle (5pt);
    \fill[blue] (40, 50) circle (5pt);
    \fill[blue] (25, 60) circle (5pt);
    \fill[blue] (55, 60) circle (5pt);
    \fill[red] (50, 80) circle (5pt);
    \fill[red] (45, 80) circle (5pt);
    \fill[red] (65, 90) circle (5pt);
    \fill[red] (25, 40) circle (5pt);
    \fill[red] (55, 70) circle (5pt);
    \fill[red] (50, 90) circle (5pt);
    \fill[red] (50, 60) circle (5pt);
    \fill[red] (45, 60) circle (5pt);
    \fill[red] (65, 60) circle (5pt);
    \fill[red] (75, 20) circle (5pt);
    \fill[red] (70, 30) circle (5pt);
    \fill[red] (30, 80) circle (5pt);
    \fill[red] (40, 70) circle (5pt);
    \fill[red] (65, 30) circle (5pt);
    \fill[red] (55, 40) circle (5pt);
  \end{axis}
\end{tikzpicture}


\end{document}